\section{Diskussion}
\label{sec:Diskussion}
Dieser Versuch hat viele potentielle Fehlerquellen. Zum Einen ist die Isolierung der Stäbe nicht perfekt, weshalb hier einige Wärme während
des Versuches entweichen kann und das Ergebnis somit verfälscht. Außerdem ist anzumerken, dass die Messgeräte nur über eine bestimmte Abtastrate
verfügen, was die Ergebnisse weiter ungenau macht, wenngleich dieser Faktor bei einer nur grafischen Auswertung kaum ins Gewicht fällt.
Desweiteren wird die Messung durch verschieden Starttemperaturen der Thermoelemente verfälscht, welche bei einer solchen Messung nicht viel besser zu machen ist.
Die Ergebnisse der grafischen Auswertung der Temperaturverläufe sind unter Betrachtung der Literaturwerte für die Wärmeleitfähigkeit $\kappa$, zu sehen in  
\autoref{tab:werte}, zu Erwarten gewesen, besonders beim Vergleich der Steigungen der Graphen miteinander, welche alle von der jeweiligen Wärmeleitfähigkeit proportional
abhängen. Das Ergebnis, dass Aluminium das am besten wärmeleitende Material ist, ist ebenfalls bezogen auf die Literaturwerte und im Vergleich mit den Graphen
zu erwarten, anzumerken ist jedoch, dass sich mit dieser statischen Methode so nicht bestimmen lässt wie viel besser die Wärmeleitung in Aluminium
tatsächlich ist im Vergleich zu den anderen Materialien. Konkrete Werte sind nach dieser Methode nicht zu bestimmen.
Bei dem Vergleich der Differenzen der beiden jeweiligen Thermoelemente von dem breiten Messingstab ist hier ein grober Fehler unterlaufen.
Die beiden gezeigten Graphen sind identisch, obwohl die Werte in \autoref{tab:wstrom} nahelegen, dass die Differenzen des Edelstahlstabes wesentlich größer
sein müssen, als die des Messingstabes. Unter der Annahme, dass zumindest die Form des Graphen korrekt ist, ist anzunehmen, dass beide Verläufe zunächst steil ansteigen, ein Maximum
durchlaufen und danach abflachen und konvergieren, wobei die Differenz des Edelstahlstabes gegen einen größeren Wert konvergiert.
Dies lässt darauf schließen, dass sich die Wärme auf dem Edelstahlstab deutlich schlechter verteilt und Edelstahl somit eine deutlich geringere Wärmeleitfähigkeit
als Messing haben muss.

Die Messung nach der dynamischen Methode ist neben den oben genannten Fehlern, weiteren Ungenauigkeiten unterworfen. Hinzu kommt einerseits die manuelle Erzeugung
der Temperaturwelle durch Umlegen eines Schalters nach einer bestimmten Periodendauer, welche durch zeitliche Ungenauigkeiten und Grenzen der von Menschen machbaren 
Genauigkeiten verfälscht wird. Andererseits spielt die Funktionsweise des Peltierelements eine weitere Rolle, dessen COOL-Funktion schwächer ist und somit ein Ansteigen 
der Temperaturwelle zu Folge hat. Bei der Bestimmung der Wärmeleitfähigkeit $\kappa$ ist die vorherige Bestimmung der Amplitudenverhältnisse und der Phasendifferenz
der Temperaturwelle, welche grafisch vorgenommen wird und daher große Messungenauigkeiten hat, bedingt durch meschliche Genauigkeit und der Genauigkeit der Messinstrumte und
der Grafiken selbst, welche, wie zuvor schon erwähnt, ebenfalls nicht genau sein können. 
Die erechneten Werte kommen trotz großen Fehlers nah an die Literaturwerte \autoref{tab:werte} heran, obwohl der Wert für Aluminium schon etwas stärker abweicht.
Dies ist wohl auf einen zufälligen Fehler zurückzuführen, da dieselbe Messung bei Messing einen vätergleichsweise sehr guten Wert ergibt. 
Bei Edelstahl ist noch anzumerken, dass die Messung bereits nach drei Perioden wegen Überhitzung eines Thermoelementes abgebrochen werden musste, weshalb
diese Messung trotz gutem Ergebnis, wohl deutlich weniger repräsentativ ist, als die anderen beiden Werte.
Allgemein lässt sich nur eine geringe Repräsentativit für die berechneten Werte feststellen; hier müssten gegebenenfalls weiter Messreihen
folgen um eine höhere Genauigkeit zu erzielen.
Für alle Wärmeleitfähigkeiten lässt sich aber sagen, dass die Werte unter Berücksichtigung aller Fehlerquellen im Rahmen der Messgenauigkeit liegen.