\section{Auswertung}
\label{sec:Auswertung}
\subsection{Statische Methode}
Die Graphen, welche die Tempereaturverläufe der vier Stäbe an den Thermoelemten, welche weiter vom Peltierelement entfernt sind, darstellen finden
sich im Anhang als Graphik 1 und 2. Anzumerken sei, dass die für diese Messreihe relevanten Daten, erst ab der Markierung zu betrachten sind. Die vorherigen
Daten stellen Messschwierigkeiten dar, welche für die Ergebnisse allerdings keine Auswirkungen haben. Alle Graphen zeigen einen zuerst exponentiellen 
Anstieg, der aber mit zunehmender Zeit deutlich abflacht. Auffällig ist die Messung des Thermoelements $T_8$, also dem des Edelstahls. Die Temperatur 
steigt hier im Vergleich zu den anderen Verläufen deutlich langsamer und erreicht in der Messperiode auch deutlich geringere Temperaturen (ca. 37°C), als die 
anderen Thermoelemente (ca. 46-52°C). Die nach 700s gemessenen Temperaturen der äußeren Thermoelemente
\\ \\
\centerline{$T_1 = 321.80 \symup{K}$}
\centerline{$T_4 =  319.06 \symup{K}$}
\centerline{$T_5 =  324.31 \symup{K}$}
\centerline{$T_8 =  310.10 \symup{K}$}
\\ \\
dienen dem qualitativen Vergleich der Wärmeleitfähigkeit der unterschiedlichen Materialien.
Wie zu sehen weist das Thermoelement $T_5$ die höchste Temperatur auf. Da die Messung mit ungefähr gleichen Temperaturen an allen Stäben gestartet wurde,
lässt sich schließen, dass Aluminium von allen drei Materialien die höchste Wärmeleitfähigkeit hat. 
Zur Bestimmung des Wärmestroms werden die gemessenen Temperaturen, welche in \autoref{tab:temps} dargestellt sind, verwendet. 
Die Berechnung erfolgt dann über Gleichung \eqref{eqn:waermestrom}, wobei $\partial T$ die Änderung der Temperatur am jeweiligen Stab, das heißt 
die Temperaturdifferenz der beiden Thermoelemente am Stab, darstellt, $\kappa$ die Wärmeleitfähigkeit des Material, $A$ die Querschnittsfläche des Stabes und $\Delta x$ der Abstand zwischen den Thermoelementen ist.
Die Werte für $A$ sind der Versuchsanleitung \cite[2]{v204} und die Werte für $\kappa$ der Literatur \cite{internet} entnommen. Der Abstand $\Delta x$ ist gemessen worden und beträgt bei allen vier Stäben
$\Delta x = 0.032$m. 
Die verwendenten Werte für $A$ und $\kappa$ sind in \autoref{tab:werte} zu sehen.
Damit ergeben sich nach Gleichung \eqref{eqn:waermestrom} für den Wärmestrom die Werte welche in \autoref{tab:wstrom} zu sehen sind.
\begin{table}[!htp]
\centering
\caption{Wärmeleitfähigkeit $\kappa$ und Querschnitt $A$ verschiedener Materialien.}
\label{tab:werte}
\begin{tabular}{c c c}
\toprule
{Material} & {$\kappa / \frac{W}{mK}$} & {$A / m^2$}  \\
\midrule
Messing, breit & $113$ & $4.8 \cdot 10^{-5}$ \\
Messing, schmal & $113$ & $2.8 \cdot 10^{-5}$ \\
Aluminium & $235$ & $4.8 \cdot 10^{-5}$ \\
Edelstahl & $20$ & $4.8 \cdot 10^{-5}$ \\
\bottomrule
\end{tabular}
\end{table}
\begin{table}[!htp]
\centering
\caption{Die verwendeten Temperaturen der Thermoelemte.}
\label{tab:temps}
\begin{tabular}{S[table-format=3.0] S[table-format=2.2] S[table-format=2.2] S[table-format=2.2] S[table-format=2.2] S[table-format=2.2] S[table-format=2.2] S[table-format=2.2] S[table-format=2.2] S[table-format=2.2] }
\toprule
{$t/s$} & {$T_1/°C$} & {$T_2/°C$} & {$T_3/°C$} & {$T_4/°C$} & {$T_5/°C$} & {$T_6/°C$} & {$T_7/°C$} & {$T_8/°C$} \\
\midrule
140 & 36.12 & 40.57 & 40.94 & 35.53 & 41.07 & 43.70 & 38.30 & 26.44 \\
280 & 41.93 & 44.81 & 44.19 & 40.11 & 45.49 & 47.46 & 41.34 & 30.34 \\
420 & 44.87 & 47.33 & 46.33 & 42.52 & 47.74 & 49.36 & 43.43 & 32.97 \\
560 & 46.89 & 49.23 & 48.08 & 44.30 & 49.55 & 51.13 & 45.27 & 35.04 \\
700 & 48.65 & 50.95 & 49.67 & 45.91 & 51.16 & 52.71 & 46.99 & 36.95 \\
\bottomrule
\end{tabular}
\end{table}
\begin{table}[!htp]
\centering
\caption{Temperaturdifferenzen und Wärmestrom der Stäbe.}
\label{tab:wstrom}
\begin{tabular}{c c c c c c c c c}
\toprule
 & \multicolumn{2}{c}{Messing, breit} & \multicolumn{2}{c}{Messing, schmal} & \multicolumn{2}{c}{Aluminium} & \multicolumn{2}{c}{Edelstahl} \\
\cmidrule(lr){2-3}\cmidrule(lr){4-5}\cmidrule(lr){6-7}\cmidrule(lr){8-9}
{$t/s$}
 & {$T_2 -T_1 / K$} & {$\frac{\partial Q}{\partial t} / W$} &{$T_3 -T_4 / K$} & {$\frac{\partial Q}{\partial t} / W$}&{$T_6 -T_5 / K$} & {$\frac{\partial Q}{\partial t} / W$}&{$T_7 -T_8 / K$} & {$\frac{\partial Q}{\partial t} / W$} \\
\midrule
140 & 4.45 & -0.75 & 5.41 & -0.53 & 2.63 &-0.93&11.86&-0.36\\
280 & 2.88 & -0.49 & 4.08 & -0.40 & 1.97 &-0.69&11.00&-0.33 \\
420 & 2.46 & -0.42 & 3.81 & -0.38 & 1.62 &-0.57&10.46&-0.31\\
560 & 2.34 & -0.40 & 3.78 & -0.37 & 1.58 &-0.56&10.23&-0.31\\
700 & 2.30 & -0.39 & 3.76 & -0.37 & 1.55 &-0.55&10.04&-0.30\\
\bottomrule
\end{tabular}
\end{table}

Im Folgenden werden die Temperaturdifferenzen des breiten Messingstabes, also $T_2-T_1$, und die des Edelstahlstabes, also $T_7 - T_8$ betrachtet. Die 
zugehörigen Graphen finden sich jeweils in Grafik 3 und 4 im Anhang, wobei auch bei diesen Graphiken erst die Daten ab der Markierung zu berücksichtigen sind.
Wie zu sehen ist steigen die Kurven zunächst stark an, durchlaufen ein Maximum und flachen danach ab und scheinen dann gegen einen Wert zu konvergieren.
Zu Bemerken ist hier allerdings, dass sich die beiden Grafiken gar nicht unterscheiden, womit unter Berücksichtigung der Werte aus \autoref{tab:wstrom}, bei denen 
ein großer Unterschied zwischen den Differenzen festzustellen ist, festzustellen ist, dass beim Erstellen der Grafiken ein grober Fehler unterlaufen ist.
Näheres wird hierzu in der Diskussion weiter erklärt.

\subsection{Dynamische Methode}
Mittels der sogenannten Angströng-Methode wird im folgenden die Wärmeleitfähigkeit $\kappa$ für die drei Materialien bestimmt.
Hierzu wurden zunächst die die Temperaturwellen der beiden Thermoelemente des jeweiligen Stabes gegeneinander aufgetragen und die Amplituden $A_1,A_2$ und die Phasendifferenz $\Delta t$ durch diese Graphen bestimmt.
Die Temperaturwellen für Messing ist in Grafik 5, die für Aluminium in Grafik 6 und die für Edelstahl in Grafik 7 im Anhang zu sehen. Der hellere Graph stellt in Grafik 5 und 6
die Messung des weiter entfernten Thermoelements dar, der dunklere Graph den des Näheren. Die Belegung ist in Grafik 7 umgekehrt.
Aus den abgelesenen Amplitudenverhältnissen und Phasendifferenzen wird mittels Gleichung \eqref{eqn:mittelwert} der Mittelwert bestimmt, dessen Fehler sich mit Gleichung \eqref{eqn:FehlerMittelwert} errechnet.
Im Falle des Messingstabes sind die hierzu verwendeten Werte in \autoref{tab:messingmw} zu sehen welche der Grafik 5 entnommen sind.
Die für die Berechnung notwendigen Materialkonstanten sind in \autoref{tab:materialkonst} dargestellt.
Somit ergeben sich für die Mittelwerte:
\\ \\
\centerline{$\frac{A_\text{nah}}{A_\text{fern}} = 2.46 \pm 0.13$}
\centerline{$\Delta t = (15.4 \pm 0.7) \symup{s}$}.
\\ \\
Nach Gleichung \eqref{eqn:leitfaehigkeit} erechnet sich die Wärmeleitfähigkeit $\kappa$, der zugehörige Fehler mittels Gleichung \eqref{eqn:fehlerfortpflanzung}
und ist somit 
\begin{equation}
\label{eqn:fehlerkappa}
\Delta \kappa = \sqrt{\biggl (\frac{\rho c (\Delta x)^2}{-2 \ln{\frac{A_\text{nah}}{A_\text{fern}}} (\Delta t)^2 }\biggr)^2(dt)^2 + \biggl(\frac{\rho c (\Delta x)^2}{-2 \frac{A_\text{nah}}{A_\text{fern}} \ln{\frac{A_\text{nah}}{A_\text{fern}}}^2 (\Delta t)} \biggr )^2 (d\frac{A_\text{nah}}{A_\text{fern}})^2 }.
\end{equation}
Für Messing ergibt sich damit eine Wärmeleitfähigkeit von
\\ \\
\centerline{$\kappa_\text{Messing} = (121 \pm 9) \symup{\frac{W}{mK}}$}.
\\ \\
Dieselbe Rechnung erfolgt analog für Aluminium.
Unter Berücksichtigung der, der Grafik 6 entnommenen Werte, welche in \autoref{tab:alumw} aufgelistet sind, und den Materialkonstanten von Aluminium (siehe \autoref{tab:materialkonst})
ergeben sich für Aluminium die Mittelwerte
\\ \\
\centerline{$\frac{A_\text{nah}}{A_\text{fern}} = 1.81 \pm 0.06$}
und
\centerline{$\Delta t = (10.7 \pm 0.7) \symup{s}$}
\\ \\
und daher mittels Gleichung \eqref{eqn:leitfaehigkeit} und \eqref{eqn:fehlerkappa} die Wärmeleitfähigkeit von Aluminium
\\ \\ 
\centerline{$\kappa_\text{Aluminium} = (187 \pm 17) \symup{\frac{W}{mK}}$}.
\\ \\
Für die Bestimmung der Wärmeleitfähigkeit von Aluminium und Messing ist mit einer Temperaturwelle von 80s Periodendauer zu arbeiten,
während für die folgende Bestimmung der Wärmeleitfähigkeit von Edelstahl eine Periodendauer von 200s notwendig ist; die Berechnung erfolgt
dennoch komplett analog zu Aluminium und Messing.
Mithilfe der Gleichungen \eqref{eqn:mittelwert} und \eqref{eqn:FehlerMittelwert} erechnet sich mit den Werten in \autoref{tab:edelmw} und den Materialkonstanten die Mittelwerte
\\ \\
\centerline{$\frac{A_\text{nah}}{A_\text{fern}} = 4.00 \pm 0.15$}
und
\centerline{$\Delta t = (66.7 \pm 4.3) \symup{s}$}.
\\ \\
Somit ist die Wärmeleitfähigkeit von Edelstahl nach Gleichung \eqref{eqn:leitfaehigkeit} und \eqref{eqn:fehlerkappa}
\\ \\ 
\centerline{$\kappa_\text{Edelstahl} = (17.7 \pm 1.2) \symup{\frac{W}{mK}}$}.
\\ \\

\begin{table}[!htp]
\centering
\caption{Dichte und spezifische Wärme der verwendeten Materialien \cite[2]{V204}}
\label{tab:materialkonst}
\begin{tabular}{c c c}
\toprule
{Material} & { Dichte $\rho / \frac{kg}{m^3}$} & { spezifische Wärme $c / \frac{J}{kgK}$} \\
\midrule
Messing & 8520 & 385 \\
Aluminium & 2800 & 830 \\
Edelstahl & 8000 & 400 \\
\bottomrule
\end{tabular}
\end{table}
\begin{table}[!htp]
\centering
\caption{Die verwendeten Amplituden und Phasendifferenzen zu Messing.}
\label{tab:messingmw}
\begin{tabular}{S[table-format=1.1] S[table-format=1.1] S[table-format=2.0] }
\toprule
{$A_\text{nah} /K$} & {$A_\text{fern} /K$} & {$\Delta t /s$} \\
\midrule
8.0 & 5.4 & 17 \\
7.4 & 3.4 & 14 \\
7.6 & 3.2 & 19 \\
7.6 & 3.2 & 17 \\
7.2 & 2.8 & 14 \\
7.0 & 2.8 & 14 \\
6.8 & 2.6 & 17 \\
6.8 & 2.6 & 17 \\
6.6 & 2.2 & 11 \\
6.4 & 2.2 & 14 \\
\bottomrule
\end{tabular}
\end{table}
\begin{table}[!htp]
\centering
\caption{Die verwendeten Amplituden und Phasendifferenzen zu Aluminium.}
\label{tab:alumw}
\begin{tabular}{S[table-format=1.1] S[table-format=1.1] S[table-format=2.0] }
\toprule
{$A_\text{nah} /K$} & {$A_\text{fern} /K$} & {$\Delta t /s$} \\
\midrule
9.6 & 7.6 & 14 \\
8.8 & 4.8 & 8 \\
9.2 & 5.0 & 11 \\
8.8 & 5.0 & 11 \\
8.6 & 4.6 & 14 \\
8.6 & 4.4 & 11 \\
8.2 & 4.4 & 8 \\
8.2 & 4.4 & 11 \\
8.0 & 4.2 & 11 \\
7.8 & 4.0 & 8 \\
\bottomrule
\end{tabular}
\end{table}
\begin{table}[!htp]
\centering
\caption{Die verwendeten Amplituden und Phasendifferenzen zu Edelstahl.}
\label{tab:edelmw}
\begin{tabular}{S[table-format=2.1] S[table-format=1.1] S[table-format=2.0] }
\toprule
{$A_\text{nah} /K$} & {$A_\text{fern} /K$} & {$\Delta t /s$} \\
\midrule
14.4 & 3.8 & 74 \\
12.6 & 3.2 & 67 \\
12.0 & 2.8 & 59 \\
\bottomrule
\end{tabular}
\end{table}

