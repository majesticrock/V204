\section{Zielsetzung}
Mithilfe dieses Versuches soll die Wärmeleitfähigkeit von den Metallen Aluminium, Messing und Edelstahl untersucht werden.
Die jeweiligen für das Material spezifischen Wärmeleitfähigkeiten werden ebenfalls bestimmt.


\section{Theorie}
\label{sec:Theorie}

Herrscht in einem Material eine Temperaturdifferenz an verschiedenen Stellen, so gibt es einen Wärmeausgleich.
Dieser kann mittels Konvektion, Wärmstrahlung und Wärmeleitung stattfinden. In diesem Versuch liegt das Augenmerk auf der Wärmeleitung.
In Festkörpern erfolgt diese mittels sogenannter Phononen, das sind Schwingungen, die als Teilchen beschrieben werden können, und freien Elektronen.
In Metallen kann der Beitrag durch diese Phononen vernachlässigt werden.
Generell wird immer beobachtet, dass die Wärme von der wärmeren Stelle zu der kälteren fließt.
Dabei hängt die transferierte Wärmemenge $\symup{d}Q$ von der Geometrie des Stabes sowie den Materialeigenschaften ab. Die besagte Wärmemenge lässt sich mittels

\begin{equation}
\label{eqn:waermemenge}
    \symup{d}Q = -\kappa A \frac{\partial T}{\partial x}\symup{d}t
\end{equation}

berechnen. Hier ist $A$ die Querschnittsfläche des Stabes und $\kappa$ die spezifische Wärmeleitfähigkeit. Das negative Vorzeichen ist eine Konvention bedingt durch den oben beschrieben Fakt, dass die Wärme immer zur kälteren Stelle fließt.
Mit dieser Formel lässt sich leicht die Formel für den Wärmestrom $\frac{\symup{d}Q}{\symup{d}t}$ zeigen:

\begin{equation}
\label{eqn:waermestrom}
    \frac{\symup{d}Q}{\symup{d}t} = -\kappa A \frac{\partial T}{\partial x}.
\end{equation}

Mithilfe der Kontinuitätsgleichung 
\begin{equation}
\label{eqn:kontinuitaet}
    \frac{\partial}{\partial t}\frac{\partial Q}{\partial V} + \symup{div} \symbf{j} = 0
\end{equation}
und folgenden Beziehungen für $\partial Q$ und $\symbf{j}$

\centerline{$\symbf{j} = - \kappa \frac{\partial T}{\partial x}$}

\centerline{$\partial Q = mc \cdot \partial T$}

lässt sich die eindimensionale Diffusionsgleichung herleiten:
\begin{equation}
\label{eqn:diffusion}
    \frac{\partial T}{\partial t} = \frac{\kappa}{\rho c} \cdot \frac{\partial^2 T}{\partial x^2}    .
\end{equation}

Dabei ist $c$ die spezifische Wärmekapazität des Stoffes, $\rho$ dessen Dichte und $m$ die Masse des Körpers.

Wird der Körper periodisch aufgeheizt und abgekühlt, so lässt sich in diesem eine Temperaturwelle feststellen.
Diese wird mittels
\begin{equation}
\label{eqn:term-welle}
    T(x, t) = T_0 \cdot \exp \bigg( -\sqrt{\frac{\omega \rho c}{2 \kappa}} x \bigg) \cdot \cos\bigg( \omega t - \sqrt{\frac{\omega \rho c}{2 \kappa} x} \bigg)
\end{equation}
beschrieben. Dabei ist $T_0$ die Amplitude der Welle. Der Dämpfungsgrad wird durch die Exponentialfunktion beschrieben.
Werden Amplitude an zwei verschiedenen Stellen gemessen, lässt sich aus deren Verhältnis und der Frequenz der Welle die Dämpfung berechnen.
Dazu wird folgende Formel für die Phasengeschwindigkeit der Welle benötigt:
\begin{equation}
\label{eqn:phasengeschwindigkeit}
    v = \sqrt{\frac{2 \omega \kappa}{\rho c}}  .
\end{equation}
Werden nun die Verhältnisse 

\centerline{$\omega = \frac{2 \pi}{t_\text{Periode}}$}

und

\centerline{$\phi = \frac{2 \pi \Delta t}{t_\text{Periode}}$}

genutzt, wobei $t_\text{Periode}$ die Periodendauer und $\phi$ die Phasenverschiebung ist, lässt sich die Formel
\begin{equation}
\label{eqn:leitfaehigkeit}
    \kappa = \frac{\rho c (\Delta x)^2}{2 \Delta t \ln \Big(\frac{A_\text{nah}}{A_\text{fern}}\Big)}
\end{equation}
berechnen. Dabei sind $A_\text{nah}$ und $A_\text{fern}$ die beiden gemessenen Amplituden.