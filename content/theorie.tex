\section{Zielsetzung}
Mithilfe dieses Versuches soll die Wärmeleitfähigkeit von den Metallen Aluminium, Messing und Edelstahl untersucht werden.
Die jeweiligen für das Material spezifischen Wärmeleitfähigkeiten werden ebenfalls bestimmt.


\section{Theorie}
\label{sec:Theorie}

Herrscht in einem Material eine Temperaturdiffrenz an verschiedenen Stellen, so gibt es einen Wärmeausgleich.
Dieser kann mittels Konvektion, Wärmstrahlung und Wärmeleitung stattfinden. In diesem Versuch liegt das Augenmerk auf der Wärmeleitung.
Generell wird immer beobachtet, dass die Wärme von der wärmeren Stelle zu der kälteren fließt.
Dabei hängt die transferierte Wärmemenge $\symup{d}Q$ von der Geometrie des Stabes sowie den Materialeigenschaften ab. Die besagte Wärmemenge lässt sich mittels

\begin{equation}
\label{eqn:waermemenge}
    \symup{d}Q = -\kappa A \frac{\partial T}{\partial x}\symup{d}t
\end{equation}

berechnen. Hier ist $A$ die Querschnittsfläche des Stabes und $\kappa$ die spezifische Wärmeleitfähigkeit. Das negative Vorzeichen ist eine Konvention bedingt durch den oben beschrieben Fakt, dass die Wärme immer zur kälteren Stelle fließt.
